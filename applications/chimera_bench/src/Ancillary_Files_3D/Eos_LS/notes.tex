%-----------------------------------------------------------
%
%     Note: this is a LATEX file that is self contained.
%     If you can't LATEX this at your site let me know and
%     I will mail you a hardcopy.
%
%-----------------------------------------------------------


\documentstyle[]{article}
\textheight=21cm
\textwidth=15.5cm
\voffset=-.5in
\hoffset=-1cm
\begin{document}
\baselineskip=14pt
\newcounter{lit}
\setcounter{lit}{0}
%
%
%
\bf
\vskip 1.0cm
\centerline {USER NOTES FOR THE EQUATION OF STATE}
\vskip 0.75cm
\centerline {F. Douglas Swesty}
\it
\centerline {Department of Physics, State University of New York at Stony 
Brook,}
\centerline {Stony Brook, NY 11794}
\vskip 2.5cm
%abstract goes here
%\small
\bf
\centerline {Contents}
\rm
\vskip 0.25cm

\begin{itemize}
\item 1. Introduction

\item 2. Caveats

\item 3. Initializing the EOS

\item 4. Boundary and Maxwell Construction Files

\item 5. Calling the EOS

\item 6. Accessing the EOS variables

\item 7. How to change the Electron EOS to your favorite version

\item 8. Tips for users

\item 9. Reporting suspected problems

\end{itemize}
%
%

\normalsize
\rm
\pagebreak
% here starts the main body of the text
\baselineskip=14pt


\vskip 0.3in
\bf
\centerline{1. Introduction}
\rm
  This document is designed to serve as a users guide to the EOS code
of Lattimer and Swesty \cite{ls}, \cite{llpr}.  
Both the formulation of the EOS and
the code itself were designed specifically to be used in radiation 
hydrodynamic simulations.  The code and the associated common blocks
are rather lengthy, and I would recommend against making any changes
to it, since you may run  into problems.  If you have to make changes to
get the code to run on your machine, please proceed cautiously.
The EOS code is divided into several main parts.  There are two main
sub-EOS's: the nuclear EOS (where nuclei exist) and the bulk EOS where
there are only alpha particles and exterior nucleons.  There is a
module that combines these two EOS's in the Maxwell construction regime.
And finally, there is a logic module that determines which EOS to use.
These things are all transparent to the user, but I will refer to them
in the following sections.

\vskip 0.3in
\bf
\centerline{2. Caveats}
\rm

  The code presently has a few limitations in what in can do.  
First and foremost it is not idiot proof.  If you give it garbage inputs
it will undoubtedly crash or at least give you garbage back.  I
have not put any checks in to make sure the inputs are reasonable
since this would merely cost more CPU time.  The responsibility
for making sure the inputs are reasonable is yours.

  Secondly, all inputs are in nuclear units, i.e., Mev \& fermis.  For specific
information on individual variables see Section 6 of these notes.

  The EOS code currently is restricted to the following ranges
of temperature, density and Ye: $T>0.05$ MeV, $n > 10^{-8}$ per cubic fermi,
and $ 0.55> Y_e > 0.03$.  There is the slight possibility that you may have
a few problems with the EOS when you are in a situation with low temperatures
and low $Y_e$'s (i.e. in $\beta$ equilibrium at low $T$).  We therefore
recommend at present that all $\beta$ equilibrium calculations be done at
temperatures of greater than 0.5 MeV.  We hope to remove even this constraint
from the EOS code in the near future.

  Lastly, remember that this EOS assumes the condition of nuclear statistical
equilibrium.  If you try to apply this EOS to a non-NSE situation you may run 
into problems.  Specifically, I am thinking of the situation where you may
start with a precollapse core containing the silicon shell as well as the iron.
If you
start with such an initial configuration that was computed with a network, 
and you use the density, electron fraction, and temperature or internal
energy as inputs to our EOS the pressures will have an excess or deficit
with respect to those computed with the network.  Typically, I have found
these differences to be small enough that I can live with them.
Alternatively, you can adjust the initial temperature or internal energy
so that the two pressures match, or visa versa.

\vskip 0.3in
\bf
\centerline{3. Initializing the EOS}
\rm

  In order to use the EOS a call must be made to the initialization routine
that loads all the tables the EOS needs.  This call should only be made once
before you call the EOS for the first time.  The call will simply be:

\centerline{ CALL LOADMX() }

  This routine requires that the file ``fermi.atb'' be present in the working
directory.  DO NOT TAMPER THIS FILE.
This file contains the cubic spline fit coefficients for the Fermi
integrals and will be read in automatically by the initialization routine.
Additionally, at run time the user will be prompted for two other files
containing the phase and Maxwell construction boundaries that must be supplied.
These file are discussed in the following section.

\vskip 0.3in
\bf
\centerline{4. Boundary and Maxwell Construction Files}
\rm

  The call to the LOADMX routine will cause the user to be prompted for
files containing the loci of the phase and Maxwell construction boundaries
(for further discussion see our preprint).  These files come in pairs; one pair
for each set of nuclear force parameters.  As with the fermi.atb file, there is
no reason that you should alter these files.  At present we have sets of files
for three different compressibilities:  $K = 180,220,$ and $375$ MeV.  These
files are appropriately named bdXXX.atb and maxXXX.atb where XXX is 180, 220,
or 375.  The bdXXX.atb file contains the phase boundary loci, and maxXXX.atb
contains the Maxwell construction loci.  The initialization routine is smart 
enough that it will not let you use a phase boundary and a Maxwell 
construction boundary that were computed with
different sets of nuclear force parameters.
  The code that actually produces the boundary and Maxwell construction tables
is kind of finicky in that it needs fairly accurate initial guesses to get
started, and it is not very user friendly.  
For this reason I am not circulating it at present.  If you want tables for
alternate parameter sets just let me know, and I'll crank them out (please
don't ask for tables for fifty different parameter sets; each set
takes me about 8 hours of CPU time to build 
on a fast RISC architecture machine).  In the future I plan on cranking
out a bunch of tables for various parameter sets, and I will let you know
when I have them.




\vskip 0.3in
\bf
\centerline{5. Calling the EOS}
\rm
  The call to the EOS in your FORTRAN code should look like: \\
\vskip 0.2in
 CALL INVEOS(INPVAR,TOLD,YE,BRYDNS,IFLAG,EOSFLG,FORFLG,
SF,XPREV,PPREV) \\
\vskip 0.2in
The parameters are as follows:

\begin{itemize}

\item INPVAR ({\bf Declared: DOUBLE PRECISION INPVAR(4)}) contains the input
values for the EOS.  INPVAR(1) is either the temperature, the entropy per
baryon, or the internal energy per baryon (IFLAG must be set to indicate
which of these quantities you are using).  If you are inputting the temperature
or the internal energy per baryon, it must be in units of MeV.  
If you are inputting the entropy per baryon, it must be in units of Boltzmanns
constant.
INPVAR(2), INPVAR(3), and INPVAR(4) are initial guesses at the solution of the
equilibrium equations.
INPVAR(2) is the number density of nucleons in nuclei (in units
of fm$^{-3}$).  Typically a good initial guess for this is $n_i \sim 0.155$.
INPVAR(3) is $\eta_{po}$.  A good initial guess for this is $\eta_{po} = -15$.
Similarly, INPVAR(4) is $\eta_{no}$, and a good initial guess for this is
$\eta_{no} = -10$.  INPVAR(2), INPVAR(3), and INPVAR(4) return the true
values of these variables after solving the equilibrium equations, so it is
a good idea to save these for each zone where the EOS is called to use as 
initial guesses on the next call.  Using the previous values allows a much
faster solution to the equilibrium equations then would otherwise be possible.

\item TOLD ({\bf Declared: DOUBLE PRECISION TOLD}) This is the initial guess at
the temperature in units of MeV.  This variable is only used by the EOS if
you are inputting internal energy per baryon or entropy per baryon via
INPVAR(1).  In those cases, the true value of the temperature is returned
by the EOS in TOLD.  Again, it is a good idea to save this for use as an
initial guess on the next call.  In the case where you are inputting the 
temperature via INPVAR(1) you may just ignore this variable altogether.

\item YE
({\bf Declared: DOUBLE PRECISION YE}) This is the net electron fraction
which the EOS needs.

\item BRYDNS ({\bf Declared: DOUBLE PRECISION BRYDNS}) This is the baryon number
density which you must supply in units of per cubic fermi.

\item IFLAG ({\bf Declared: INTEGER IFLAG}) 
This variable tells the EOS whether you
are inputting temperature, internal energy per baryon, or entropy per baryon.
Set IFLAG to 1 to indicate temperature, 2 to indicate internal energy per 
baryon, and 3 to indicate entropy per baryon.

\item EOSFLG ({\bf Declared: INTEGER EOSFLG}) This variable returns a value 
which
tells you what region of the temperature--density plane you are in.  A value
of 2 indicates the nuclear EOS was used.  A value of 3 indicates the Bulk EOS
(no nuclei) was used.  A value of 4 indicates that you are in the Maxwell 
construction region.  It is not necessary to set this variable prior to the
call UNLESS you specifically want to force INVEOS to use a specific EOS
(see FORFLG).  I would highly recommend against this as you may be trying to use
a specific EOS in a region where the equilibrium equations have no solution.  
Otherwise, the EOS has a
logic module to determine what region of the temperature density plane it is in
and it automatically sets EOSFLG accordingly.

\item FORFLG ({\bf Declared: INTEGER FORFLG}) This is the forcing flag.  
If FORFLG
is set to zero, the code automatically figures out which region of the 
temperature-density plane it is in, and sets EOSFLG accordingly.  If FORFLG
is set to 1, the code uses the EOS that you indicate by setting EOSFLG prior
to the call to INVEOS.  I would recommend against using the latter option, as 
it was intended for my own debugging purposes.

\item SF ({\bf Declared: INTEGER SF})  This variable is the success flag.
A returned value of 1 indicates a successful EOS call; anything else
indicates a failure of some sort.  In the case of a failure, the bulk EOS is
called and the thermodynamic variables from it are returned (this is an 
artifice designed to keep the calling code from crashing).  If you are going
to have trouble with the EOS, a tripped success flag will almost 
certainly be the first indicator.  Thus I would recommend always checking
this flag after each call.


\item XPREV ({\bf Declared: DOUBLE PRECISION XPREV})  
Ignore this variable by simply
sticking in a dummy variable in the call.  It is used for my own debugging
purposes.

\item PPREV ({\bf Declared: DOUBLE PRECISION PPREV}) This variable holds the
initial guess at the exterior proton fraction and is required by INVEOS.
When there are no nuclei present this is the variable iterated on to solve
the equilibrium equations.  A good initial guess is YE*BRYDNS.  The true
value is returned by INVEOS, and it would be a good idea to save this for the
next call.


\end{itemize}

  All the EOS variables that you may want for your own purposes (pressures,
entropies, chemical potentials, etc.) are passed back to the calling code via
common blocks.  This is described in the next section.



\vskip 0.3in
\bf
\centerline{6. Accessing the EOS variables}
\rm

  All the information you will probably want and much more is passed back to
the calling routines through the common blocks in the INCLUDE file
``eos\_m4a.inc''.  I would recommend using this include file in your code
as opposed to coding in the common blocks by hand.  If I happen to change
the structure of the common blocks in the future, this will let you
avoid unnecessary changes to your code.  If you find that my variable
definitions conflict with yours I WOULD HIGHLY RECOMMEND WRITING A LITTLE
``WRAPPER'' SUBROUTINE THAT CONVERTS MY VARIABLES TO YOURS AS OPPOSED TO 
CHANGING MY VARIABLE NAMES OR COMMON BLOCKS (DOING SO COULD QUICKLY GET YOU 
INTO AN UNBELIEVABLE MESS).  The ``wrapper'' routine can
then be called directly from your code.

  The common blocks and their variables are described below.  NOTE THAT THERE IS
NO IMPLICIT TYPING SCHEME PRESENT IN MY CODE!!!!  DON'T ASSUME THAT A VARIABLE IS
INTEGER OR DOUBLE PRECISION BECAUSE OF ITS SPELLING!!!  CHECK THE INCLUDE FILE
TO MAKE SURE.  ALSO, NOTE THAT EVERY FLOATING POINT VARIABLE IN THE CODE IS
DOUBLE PRECISION; THERE ARE NO ``REAL'' VARIABLES.

  Basically, everything you want will be in the following three common blocks.
The other common blocks largely contain EOS internals that will be of no use to
you in your codes.  If there is a quantity you need and it is not listed here
let me know and if it is not in one of the other common blocks already, I will
put it in.

\centerline{ Common block /EOSVAR/ }
\begin{itemize}

\item MUN.  The neutron chemical potential in MeV

\item MUPROT. The proton chemical potential in MeV

\item MUALFA. The alpha particle chemical potential in MeV

\item MUHAT.  MUN-MUPROT in MeV

\item BU.  Baryon internal energy per baryon in MeV

\item UTOT. Total internal energy per baryon (including electrons and
photons) in units of MeV

\item BS.  Baryon entropy per baryon in units of Boltzmann's constant

\item STOT.  Total entropy per baryon (including electrons and
photons) in units of Boltzmann's constant

\item BPRESS. Baryon pressure in units of MeV per cubic fermi

\item PTOT.  Total pressure (including electrons and photons)
in units of MeV per cubic fermi

\item XNUT.  Mass fraction of exterior neutrons

\item XPROT. Mass fraction of exterior protons

\item XALFA. Mass fraction of alpha particles

\item XH.  Mass fraction of heavy nuclei

\item X.  Internal proton fraction of nuclei ($=Z/A$)

\item A.  Mass number of heavy nuclei

\end{itemize}


\centerline{ Common block /DERIVS/ }
\begin{itemize}

\item Note: ``density'' here means the baryon number density, $n$

\item DUDT.  Derivative of total internal energy per baryon w.r.t. temperature 
(per baryon)

\item DUDN.  Derivative of total internal energy per baryon w.r.t. density
(MeV$\cdot fm^3$ per baryon)

\item DUDY.  Derivative of total internal energy per baryon w.r.t. $Y_e$
(MeV per baryon)

\item DSDT.  Derivative of total entropy per baryon w.r.t. temperature 
(per MeV per baryon)

\item DSDN.  Derivative of total entropy per baryon w.r.t. density
(fm$^3$ per baryon)

\item DSDY.  Derivative of total entropy per baryon w.r.t. $Y_e$
(per baryon)

\item DPDT.  Derivative of total pressure w.r.t. temperature 
(per cubic fermi)

\item DPDN.  Derivative of total pressure w.r.t. density
(MeV)

\item DPDY.  Derivative of total pressure w.r.t. $Y_e$
(MeV per cubic fermi)

\item GAM\_S.  The adiabatic index for the EOS

\end{itemize}


The following common block contains all the information from the electron and
photon EOS.  CAUTION: If you change the electron EOS 
EVERYTHING in this common block must be supplied by the electron
EOS if the code is to run, especially the derivatives.

\centerline{ Common block /ELECTR/ }
\begin{itemize}

\item NSUBE.  Net electron number density (YE*BRYDNS) (per cubic fermi)

\item MUSUBE.  Electron chemical potential (MeV)

\item EPRESS. Electron pressure (MeV per cubic fermi)

\item EU. Electron internal per baryon (MeV)

\item FSUBE. Electron free energy per baryon (MeV)

\item ES. Electron entropy per baryon (in units of Boltzmann's constant)

\item PPRESS. Photon pressure (MeV per cubic fermi)

\item PS. Photon entropy per baryon (in units of Boltzmann's constant)

\item PU. Photon internal energy per baryon (MeV)

\item PF. Photon free energy per baryon (MeV)

\item DEMUDT. Derivative of electron chem. pot. w.r.t. T (MeV per MeV)

\item DEMUDN. Derivative of electron chem. pot. w.r.t. density
(MeV$\cdot fm^3$)

\item DEMUDY. Derivative of electron chem. pot. w.r.t. $Y_e$ (MeV)

\item DEPDT. Derivative of electron pressure w.r.t. T (per cubic fermi)

\item DEPDN. Derivative of electron pressure w.r.t. density
(MeV)

\item DEPDY. Derivative of electron pressure w.r.t. $Y_e$ (MeV per cubic fermi)

\item DESDT. Derivative of electron entropy w.r.t. T (per MeV per baryon)

\item DESDN. Derivative of electron entropy w.r.t. density (fm$^3$
per baryon)

\item DESDY. Derivative of electron entropy w.r.t. $Y_e$ (per baryon)

\item DEUDT. Derivative of electron internal energy per baryon w.r.t. T 
(per baryon)

\item DEUDN. Derivative of electron internal energy per baryon w.r.t. density
(MeV$\cdot fm^3$)

\item DEUDY. Derivative of electron internal energy per baryon w.r.t. $Y_e$ 
(MeV)

\item DPPDT. Derivative of photon pressure w.r.t. T (per cubic fermi)

\item DPPDN. Derivative of photon pressure w.r.t. density
(MeV)

\item DPPDY. Derivative of photon pressure w.r.t. $Y_e$ (MeV per cubic fermi)
(this quantity is zero)

\item DPSDT. Derivative of photon entropy w.r.t. T (per MeV per baryon)

\item DPSDN. Derivative of photon entropy w.r.t. density (fm$^3$
per baryon)

\item DPSDY. Derivative of photon entropy w.r.t. $Y_e$ (per baryon)
(this quantity is actually zero)

\item DPUDT. Derivative of photon internal energy per baryon w.r.t. T 
(per baryon)

\item DPUDN. Derivative of photon internal energy per baryon w.r.t. density
(MeV$\cdot fm^3$)

\item DPUDY. Derivative of photon internal energy per baryon w.r.t. $Y_e$ 
(MeV) (this quantity is actually zero)

\end{itemize}



\vskip 0.3in
\bf
\centerline{7. How to change the Electron EOS to your favorite version}
\rm

  Changing the electron EOS to your personal favorite is relatively easy to do.
Simply remove the {\bf EL\_EOS} subroutine from the {\bf eos.f} file and 
replace it with your own, keeping the call statement the same.  This should be
easy since the only inputs needed are $T$, $Y_e$, and $n$.  Next ALL the 
variables in the common block {\bf \/ELECTR\/} need to be set in your new
electron EOS; especially the derivatives.  Don't forget to include the photons
also (barring plasma effects, this is only a few lines of code that you can 
steal from my routine).
Lastly, make sure that your electron EOS is thermodynamically consistent.  The
overall EOS won't be thermodynamically consistent if your electron EOS isn't.
Don't throw my electron EOS away however as you will want to put it back in
for debugging purposes if you run into a problem.


\vskip 0.3in
\bf
\centerline{8. Tips for users}
\rm

  This section will gradually expand as time goes on.  For now I want to offer
a few important tips.  
First, save the returned values (INPVAR(2:4),TOLD,PPREV) from 
the EOS to use on the next call IN THE SAME ZONE.  These values will allow
allow the Newton-Raphson to start off practically right on top of the
solution and thus will allow it to converge faster.  This will be a major
savings in CPU time right off the bat.  Secondly, in hydrocodes, when
updating the internal energy via the first law, iterate on the temperature
as opposed to the internal energy (see the discussion section of our preprint).
The EOS runs at least an order of magnitude faster when you give it the 
temperature as an input as opposed to the internal energy or entropy.
Third, use the derivatives that the EOS returns instead of calling the
EOS multiple times to take numerical derivatives.  The derivatives the EOS
returns are very accurate (far more so than numerical derivatives) and this
combined with the fact of only having to make a single EOS call represents
a significant savings of time.
Lastly, when crossing some of the phase boundaries, the derivatives of
thermodynamic variables can change abruptly (this is a real effect not
a numerical problem) so beware!  Such discontinuities can throw a serious
monkey-wrench into Newton-Raphson schemes.  In these cases (like solving the
first law when crossing the Maxwell construction region at high temperatures)
I have always be able to remedy the situation by dropping into a bisection
scheme when the Newton-Raphson fails.

  One more note for CRAY users.  If you have looked at my code by now you have
no doubt noticed it is double precision.  Rather than screw around changing
it to single precision for the CRAY, I simply use the ``-dp'' option to
CF77 when compiling.  This automatically compiles the code as single
precision and makes all the appropriate changes in intrinsic functions etc.
I have this code running on the YMP 8-32 at the Pittsburgh
Supercomputing Center in such a fashion and it seems to work fine.


\vskip 0.3in
\bf
\centerline{9. Reporting suspected problems}
\rm

  Caveat emptor. Undoubtedly, someone will find a bug or two in this code.  
Hopefully, they will be very minor ones.  I am fairly
sure I have beaten all the major bugs (the ones the have drastic effects on
radiation hydrocodes) out of this code.  However I will not stake my life on 
this. I have had this code running successfully on various VAX\/VMS machines, 
DEC ULTRIX machines, and on a CRAY YMP under UNICOS.

  Certainly, if you find a bug let me know at the address on the front page
or via internet: fswesty@sbast5.ess.sunysb.edu or bitnet: swesty@sunysbnp.
I will get on it ASAP, PROVIDED that you do not alter my code anymore than
minimally necessary to get it to compile on your machine.

  If you find what you think is a bug there are a number of things I ask you
to do.  First, if you have your own electron EOS in the code, put mine back in
and try it to see if you still get the bug.  Also, write down ALL the inputs
to the INVEOS code at the point it is failing to as many significant digits 
as your machine allows.  Then write a little dozen line driver program to try
the EOS at this point specifically.  If the code doesn't fail then chances are
your code is causing the problem by writing out-of-bounds to an array or
you have an equivalence statement screwed up, etc.  Also, check to make sure
that you have all the variables in the INVEOS call statement properly
declared as INTEGER and DOUBLE PRECISION.  If the code still fails
with a separate driving program calling it, then try turning off all the
optimization passes in the compiler and turning on bounds and overflow
checking where possible.  Especially if you are running on a UNIX
machine where the f77 compilers are notoriously bad.  
I have found a number of bugs in various UNIX compilers and recently
I found one caused by the optimizer. AFTER you have gone through all of
these steps (the object being to narrow down the place where the bug occurs) 
and haven't found the solution to the problem contact me.
If possible, try the little driver program on another
machine running a different operating system (or at least one
running a different version of FORTRAN).  This will tell us if the problem is 
machine specific.  When you contact me, give me the all inputs to INVEOS to as
many significant digits as possible, as well as every thing (error messages, 
etc.) that you know about the problem.  I will get back to you as soon as I can.

  Also, be forwarned that my code sometimes spits out error messages that look
fairly serious (I think I have removed all the obscene ones) but don't
panic.  It is quite possible that these won't even apply to your use of the 
code.  At any rate let me know if you encounter anything you are in doubt 
about.

  Lastly, I would like to apologize in advance for any problems that you may
encounter with this EOS. 
I will try to track down any problems as fast as I can, but I
would appreciate your patience in this matter as software support is not
what I am tasked with in life.

%---------------------------------------------------------------------------
%                 References
%---------------------------------------------------------------------------
%
%\pagebreak
\noindent
\baselineskip=20pt
\bf
\centerline {References}
\normalsize
\rm
\begin{thebibliography}{MM}

\bibitem[1] {ls} Lattimer, James M. and Swesty, F. Douglas, Nucl. Phys. A. \
{\it accepted for publication}, and preprint

\bibitem[2] {llpr} J. M. Lattimer, C. J. Pethick, D. G. Ravenhall, 
and D. Q. Lamb, Nucl. Phys. {\bf A432}, 646, 1985

\end{thebibliography}
\vfill
\eject

\end{document}


