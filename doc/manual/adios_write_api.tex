\chapter{ADIOS Write API}

As mentioned earlier, ADIOS writing is comprised of two parts: the XML configuration 
file and APIs. In this section, we will explain the functionality of the writing 
API in detail and how they are applied in the program.  

\section{Write API Description}

\subsection{Introduction}

ADIOS provides both Fortran and C routines. All ADIOS routines and constants begin 
with the prefix ``adios\_''. For the remainder of this section, only the C versions 
of ADIOS APIs are presented. The primary differences between the C and Fortran 
routines is that error codes are returned in a separate argument for Fortran as 
opposed to the return value for C routines. 

A unique feature of ADIOS is group implementation, which is constituted by a list 
of variables and associated with individual transport methods. This flexibility 
allows the applications to make the best use of the file system according to its 
own different I/O patterns.\label{HToc84890236}\label{HToc212016612}\label{HToc212016854}\label{HToc182553350}

\subsection{ADIOS-required functions}

This section contains the basic functions needed to integrate ADIOS into scientific 
applications. ADIOS is a lightweight I/O library, and there are only seven required 
functions from which users can write scalable, portable programs with flexible 
I/O implementation on supported platforms:

\textbf{adios\_init---}initialize ADIOS and load the configuration file

\textbf{adios\_open---}open the group associated with the file

\textbf{adios\_group\_size---}pass the group size to allocate the memory

\textbf{adios\_write---}write the data either to internal buffer or disk

\textbf{adios\_read---}associate the buffer space for data read into

\textbf{adios\_close---}commit write/read operation and close the data

\textbf{adios\_finalize---}terminate ADIOS

You can add functions to your working knowledge incrementally without having to 
learn everything at once. For example, you can achieve better I/O performance on 
some platforms by simply adding the asynchronous functions adios\_start\_calculation, 
adios\_end\_calculation, and adios\_end\_iteration to your repertoire. These functions 
will be detailed below in addition to the seven indispensable functions.

The following provides the detailed descriptions of required APIs when users apply 
ADIOS in the Fortran or C applications.

\subsubsection{adios\_init}

This API is required only once in the program. It loads XML configuration file 
and establishes the execution environment. Before any ADIOS operation starts, adios\_init 
is required to be called to create internal representations of various data types 
and to define the transport methods used for writing. 

\begin{lstlisting}[language=C]
int adios_init (const char * xml_fname)
\end{lstlisting}

Input: 
\begin{itemize}
\item xml\_fname - string containing the name of the XML configuration file
\end{itemize}

Fortran example: 
\begin{lstlisting}[language=Fortran]
call adios_init ("config.xml", ierr)
\end{lstlisting}

\subsubsection{adios\_open}

This API is called whenever a new output file is opened. adios\_open, corresponding 
to fopen (not surprisingly), opens an adios-group given by group\_name\textit{ 
}and associates it with one or a list of transport methods, which can be identified 
in future operations by the File structure whose pointer is returned as\textit{ 
}fd\_p. The group name should match the one defined in the XML file. The I/O handle 
fd\_p prepares the data types for the subsequent calls to write data using the 
io\_handle. The third argument, file\_name, is a string representing the name of 
the file. As the last argument, mode is a string containing a file access mode. 
It can be any of these three mode specifiers: ``r,'' ``w,'' or ``a.'' Currently, 
ADIOS supports three access modes: ``write or create if file does not exist,'' 
``read,'' and ``append file.'' The call opens the file only if no coordination 
is needed among processes for transport methods that the users have chosen for 
this adios\_group, such as POSIX method. Otherwise, the actual file will be opened 
in adios\_group\_size based on the provided argument comm, which will be examined 
in Sect. 4.1.2.3. As the last argument, we pass the pointer of coordination communicator 
down to the transport method layer in ADIOS. This communicator is required in MPI-IO-based 
methods such as collective and independent MPI-IO.

\begin{lstlisting}[language=C]
int adios_open (int64_t * fd_p, const char * group_name 
	,const char * file_name, const char * mode,void *comm)
\end{lstlisting}

Input: 
\begin{itemize}
\item fd\_p---pointer to the internal file structure
\item group\_name---string containing the name of the group 
\item file\_name---string containing the name of the file to be opened 
\item mode---string containing  a file access mode
\item comm--- communicator for multi-process coordination
\end{itemize}

\begin{lstlisting}[language=Fortran]
call adios_open (handle, "restart", "restart.bp", "w", comm, ierr)
\end{lstlisting}

\subsubsection{adios\_group\_size}
This function passes the size of the group to the internal ADIOS transport structure 
to facilitate the internal buffer management and to construct the group index table. 
The first argument is the file handle. The second argument is the size of the payload 
for the group opened in the adios\_open routine. This value can be calculated manually 
or through our python script. It does not affect read operation because the size 
of the data can be retrieved from the file itself. The third argument is the returned 
value for the total size of this group, including payload size and the metadata 
overhead. The value can be used for performance benchmarks, such as I/O speed. 

\begin{lstlisting}[language=C]
int adios_group_size (int64_t * fd_p, uint64_t group_size, uint64_t * total_size)
\end{lstlisting}

Input: 
\begin{itemize}
\item fd\_p---pointer to the internal file structure
\item group\_size---size of data payload in bytes to be written out. If there is an integer 
2 $\times$ 3 array, the payload size is 4 $\times$ 2 $\times$ 3 (4 is the size of integer)
\end{itemize}

output :
\begin{itemize}
\item total\_size---the total sum of payload and overhead, which includes name, data 
type, dimensions and other metadata)
\end{itemize}

Fortran example: 
\begin{lstlisting}[language=Fortran, caption={}]
call adios_group_size (handle, groupsize, totalsize, ierr)
\end{lstlisting}

\subsubsection{adios\_write}
The adios\_write routine submits a data element var for writing and associates 
it with the given var\_name, which has been defined in the adios group opened by 
adios\_open. If the ADIOS buffer is big enough to hold all the data that the adios 
group needs to write, this API only copies the data to buffer. Otherwise, adios\_write 
will write to disk without buffering. Currently, adios\_write supports only the 
address of the contiguous block of memory to be written. In the case of a noncontiguous 
array comprising a series of subcontiguous memory blocks, var should be given separately 
for each piece.

In the next XML section, we will further explain that var\_name is the value of 
attribute ``name'' while var is the value of attribute ``gwrite,'' both of which 
are defined in the corresponding \texttt{<}var\texttt{>} element inside adios\_group 
in the XML file. By default, it will be the same as the value of attribute ``name'' 
if ``gwrite'' is not defined. 

\begin{lstlisting}[language=C,caption={},label={}]
int adios_write (int64_t fd_p, const char * var_name, void * var)
\end{lstlisting}

Input:
\begin{itemize}
\item fd\_p---pointer to the internal file structure
\item var\_name---string containing the annotation name of scalar or vector in the XML file
\item var ---the address of the data element defined need to be written
\end{itemize}

Fortran example: 
\begin{lstlisting}[language=Fortran,caption={},label={}]
call adios_write (handle, "myvar", v, ierr)
\end{lstlisting}

\subsubsection{adios\_read}
\label{section:adios_read}

The write API contains a read function (historically, the first one) that uses 
the same transport method and the xml config file to read in data. It works only 
on the same number of processes as the data was written out. Typically, checkpoint/restart 
files are written and read on the same number of processors and this function is 
the simplest way to read in data. However, if you need to read in on a different 
number of processors, or you do not want to carry the xml config file with the 
reading application, you should use the newer and more generic read API discussed 
in Section 7.

Similar to adios\_write, {\color{color01} adios\_read submits a buffer space var 
for reading a data element into. This does NOT actually perform the read. Actual 
population of the buffer space will happen on the call to adios\_close. In other 
words, the value(s) of var can only be utilized after adios\_close is performed. 
Here, }var\_name corresponds to the value of attribute ``gread`` in the \texttt{<}var\texttt{>} 
element declaration while var is mapped to the value of attribute ``name.'' By 
default, it will be as same as the value of attribute ``name'' if ``gread'' is 
not defined.

\begin{lstlisting}[language=C,caption={},label={}]
int adios_read (int64_t fd_p, const char * var_name, uint64_t read_size, void * var)
\end{lstlisting}

Input:

\begin{itemize}
\item fd\_p - pointer to the internal file structure
\item var\_name - the name of variable recorded in the file
\item var - the address of variable defined in source code
\item read\_size -  size in bytes of the data to be read in 
\end{itemize}

Fortran example: 
\begin{lstlisting}[language=Fortran,caption={},label={}]
call adios_read (handle, "myvar", 8, v, ierr)
\end{lstlisting}

\subsubsection{adios\_close}

The adios\_close routine {\color{color01} commits the writing buffer to disk, closes 
the file, and releases the handle. At that point, all of the data that have been 
copied during adios\_write will be sent as-is downstream. If the handle were opened 
for read, it would fetch the data, parse it, and populate it into the provided 
buffers. This is currently hard-coded to use posix I/O calls.}

\begin{lstlisting}[language=C,caption={},label={}]
int adios_close (int64_t * fd_p);
\end{lstlisting}

Input: 
\begin{itemize}
\item fd\_p - pointer to the internal file structure
\end{itemize}

Fortran example: 
\begin{lstlisting}[language=Fortran,caption={},label={}]
call adios_close (handle, ierr)
\end{lstlisting}

\subsubsection{adios\_finalize}

The adios\_finalize routine releases all the resources allocated by ADIOS and guarantees 
that all remaining ADIOS operations are finished before the code exits. The ADIOS 
execution environment is terminated once the routine is fulfilled. The proc\_id 
parameter provides users the opportunity to customize special operation on proc\_id---usually 
the ID of the head node. 

\begin{lstlisting}[language=C,caption={},label={}]
int adios_finalize (int proc_id)
\end{lstlisting}

Input: 
\begin{itemize}
\item proc\_id - the rank of the processe in the communicator or the user-defined coordination 
variable
\end{itemize}

Fortran example: 
\begin{lstlisting}[language=Fortran,caption={},label={}]
call adios_finalize (rank, ierr)
\end{lstlisting}

call adios\_finalize (rank, ierr)\label{HToc84890237}\label{HToc212016613}\label{HToc212016855}\label{HToc182553351}

\subsection{Nonblocking functions}

\subsubsection{adios\_end\_iteration}

The adios\_end\_iteration provides the pacing indicator. Based on the entry in 
the XML file, it will tell the transport method how much time has elapsed in a 
transfer.

\subsubsection{adios\_start\_ calculation/ adios\_end\_calculation}

Together, adios\_start\_calculation and adios\_end\_calculation indicate to the 
scientific code when nonblocking methods should focus on engaging their I/O communication 
efforts because the process is mainly performing intense, stand-alone computation. 
Otherwise, the code is deemed likely to be communicating heavily for computation 
coordination. Any attempts to write or read during those times will negatively 
impact both the asynchronous I/O performance and the interprocess messaging.\label{HToc212016614}\label{HToc212016856}\label{HToc84890238}\label{HToc182553352}

\subsection{Other function}

One of our design goals is to keep ADIOS APIs as simple as possible. In addition 
to the basic I/O functions, we provide another routine listed below. 

\subsubsection{adios\_get\_write\_buffer}

The adios\_get\_write\_buffer function returns the buffer space allocated from 
the ADIOS buffer domain. In other words, instead of allocating memory from free 
memory space, users can directly use the allocated ADIOS buffer area and thus avoid 
copying memory from the ADIOS buffer to a user-defined buffer.

\begin{lstlisting}[language=C,caption={},label={}]
int adios_get_write_buffer (int64_t fd_p, const char * var_name, 
	uint64_t * size, void ** buffer)
\end{lstlisting}

Input: 
\begin{itemize}
\item fd\_p - pointer to the internal File structure
\item var\_name - name of the variable that will be read
\item size - size of the buffer to request
\end{itemize}

Output: 
\begin{itemize}
\item buffer - initial address of read-in buffer for storing the data of var\_name
\end{itemize}

\subsection{Create the first ADIOS program}

Figure 1 is a programming example that illustrates how to write a double-precision 
array t and a double-precision array with size of NX into file called ``test.bp,'' 
which is organized in BP, our native tagged binary file format. This format allows 
users to include rich metadata associated with the block of binary data as well 
the indexing mechanism for different blocks of data (see Chap. 5). 

\begin{lstlisting}[language=C,caption={ADIOS programming example.},label={list-adios-prog-example}]
/*example of parallel MPI write into a single file */ 
#include <stdio.h> // ADIOS header file required 
#include "adios.h"
int main (int argc, char *argv[])
{
	int i, rank, NX; double t [NX];
	// ADIOS variables declaration int64_t handle;
	uint_64 total_size;
	MPI_Comm comm = MPI_COMM_WORLD; 
	
	MPI_Init ( &argc, &argv);
	MPI_Comm_rank (comm, &rank);
	
	// data initialization for ( i=0; i<NX; i++)
	t [i] = i * (rank+1) + 0.1; // ADIOS routines 
	adios_init ("config.xml");
	adios_open (&handle, "temperature", "data.bp", "w",&comm); 
	adios_group_size (handle, 4, total_size);
	adios_write (handle, "NX", &NX);
	adios_write (handle, "temperature", t); 
	adios_close (handle);
	adios_finalize (rank);
	MPI_Finalize(); 
	return 0;
}
\end{lstlisting}
