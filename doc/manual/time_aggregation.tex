\chapter{Time aggregation}
\label{sec:time_aggregation}

From version 1.11, ADIOS allows aggregating multiple output steps in a memory buffer before flushing it to the file system or whatever the actual transport method is doing at \verb+adios_close()+. This temporal aggregation only works with transport methods that use the ADIOS internal buffering and thus rely on the BP file format, including the \verb+POSIX, MPI, MPI_LUSTRE, MPI_AGGREGATE, BGQ+ methods.

To use time aggregation, simply set it for a group that has been declared. The examples below show how to do that in the XML config file, or in the source code. There is a new XML element and a new function to do this, respectively. 

\begin{lstlisting}[language=XML,caption=Set time aggregation for a group in XML]
<adios-group name="diagnostics"> ... </adios-group>    
<time-aggregation group="diagnostics"   buffer-size=32000 />
\end{lstlisting}

\begin{lstlisting}[caption=Set time aggregation for a group in source code]
adios_declare_group (&groupid, ...);
adios_set_time_aggregation (groupid, 32000, NULL);
\end{lstlisting}

The buffer size is given in {\em bytes}. Make sure that at least one output step fits into the buffer. If the buffer is smaller than that, only a fraction of the output will be saved. If the buffer size is set to $0$, that turns off time aggregation for the group.


\section{Synchronizing flushes with another group's output}

One may want to enforce a policy that the time-aggregated buffer is flushed to the file system whenever a less regular I/O event happens, e.g. checkpointing. If the application dies between two checkpoints, it is guaranteed that there is data in the time-aggregated output up to the last checkpoint. One may also want a stricter policy that enforces that the aggregated output does contain steps {\em exactly} up to the last checkpoint. ADIOS provides an option for the former while the latter can be achieved by the user by allowing a buffer large enough to hold all output steps between two checkpoints.  

We can tell ADIOS that the aggregated group should be "synchronized" with another group. This will cause the buffer be flushed whenever an \verb+adios_close()+ for the {\em other group}  happens. The aggregated buffer will also be flushed whenever there is no more space in the aggregation buffer, and also at \verb+adios_finalize()+. In the XML element \verb+<time_aggregation>+, add an attribute \verb+sync_with_group="restart"+ to set the group which enforces the flushing of the aggregated group. In the non-XML source, the third parameter of the \verb+adios_set_time_aggregation+ should be the group ID of this group. 


\begin{lstlisting}[language=XML,caption=Set time aggregation for a group in XML with a synchronizing group]
<adios-group name="checkpoint"> ... </adios-group>
<adios-group name="diagnostics"> ... </adios-group>
<time-aggregation group="diagnostics" 
                  buffer-size=32000 
                  sync-with-group="checkpoint" />
\end{lstlisting}


\begin{lstlisting}[caption=Set time aggregation for a group in source code with a synchronizing group]
adios_declare_group (&ckpt_groupid, "checkpoint", "", adios_stat_default);
adios_declare_group (&diag_groupid, "diagnostics", "", adios_stat_default);
adios_set_time_aggregation (diag_groupid, 32000, ckpt_groupid);
\end{lstlisting}



