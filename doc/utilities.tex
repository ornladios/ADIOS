\vspace{34pt}
%\section*{{\Large \textbf{9 Utilities\label{HToc82067528}}}}
\section{Utilities}

\vspace{24pt}
\subsection*{{\large 9.1 }{\large \textbf{ \label{HToc84890279}\label{HToc212016655}\label{HToc212016897}\label{HToc182553426}adios\_lint}}}

\vspace{10pt}
We provide a verification tool, called adios\_lint, which comes with ADIOS. It 
can help users to eliminate unnecessary semantic errors and to verify the integrity 
of the XML file. Use of adios\_lint is very straightforward; enter the adios\_lint 
command followed by the config file name.\label{HToc82067529}\label{HToc84890280}\label{HToc212016656}\label{HToc212016898}\label{HToc182553427}

\vspace{10pt}
\subsection*{{\large 9.2 }{\large \textbf{adios\_config}}}

\vspace{10pt}
This script provides the necessary compile and linking flags to use ADIOS in your 
application and the version information of the ADIOS installation.  See Section 
2.5 for how to use it or run \texttt{"}adios\_config -h\texttt{"} to see the options. 
\label{HToc182553428}

\vspace{10pt}
\subsection*{{\large 9.3 }{\large \textbf{bpls}}}

\vspace{10pt}
The bpls utility is used to list the content of a BP file or to dump arbitrary 
subarrays of a variable. By default, it lists the variables in the file including 
the type, name, and dimensionality. Here is the description of additional options 
(use bpls \-{}h to print help on all options for this utility).

\vspace{10pt}
\leftskip=36pt
\parindent=-36pt
-l  Displays the global statistics associated with each array (minimum, maximum, 
average and standard deviation) and the value of each scalar. Note that the detailed 
listing does not have extra overhead of processing since this information is available 
in the footer of the BP file. 

\vspace{10pt}
-t  When added to the -l option, displays the statistics associated with the variables 
for every timestep. 

\vspace{10pt}
-p Dumps the histogram binning intervals and their corresponding frequencies, if 
histograms were enabled while writing the bp file. This option generates a ``\texttt{<}variable-name\texttt{>}.gpl'' 
file that can be given to the `gnuplot' program as input. 

\vspace{10pt}
-a Lists attributes besides the variables

\vspace{10pt}
-A Lists only the attributes

\vspace{10pt}
-r Sorts the full listing by names. Name masks to list only a subset of the variables/attributes 
can be given like with the -ls command or as regular expressions (with -e option).

\vspace{10pt}
-v Verbose. It prints some information about the file in the beginning before listing 
the variables. 

\vspace{10pt}
-S Dump byte arrays as strings instead of with the default numerical listing. 2D 
byte arrays are printed as a series of strings. 

\vspace{10pt}
\leftskip=0pt
\parindent=0pt
Since bpls is written in C, the order of dimensions is reported with row-major 
ordering, i.e., if Fortran application wrote an NxM 2D variable, bpls reports it 
as an MxN variable. 

\vspace{10pt}
\leftskip=36pt
\parindent=-36pt
-d Dumps the values of the variables. A subset of a variable can be dumped by using 
start and count values for each dimension with -s and -c option, e.g., -s ``10,20,30'' 
-c ``10,10,10'' reads in a 10x10x10 sub-array of a variable starting from the (10,20,30) 
element. Indices start from 0. As in Python, \-{}1 denotes the last element of 
an array and negative values are handled as counts from backward. Thus, -s ``-1,-1'' 
-c ``1,1'' reads in the very last element of a 2D array, or -s ``0,0'' -c ``1,-1'' 
reads in one row of a 2D array. Or -s ``1,1'' -c ``-2,-2'' reads in the variable 
without the edge elements (row 0, colum 0, last row and last column).

\vspace{22pt}
\leftskip=0pt
\parindent=0pt
Time is handled as an additional dimension, i.e., if a 2D variable is written several 
times into the same BP file, bpls lists it as a 3D array with the time dimension 
being the first (slowest changing) dimension. 

\vspace{10pt}
In the example below, a 4 process application wrote a 4x4 array (each process wrote 
a 2x2 subset) with values from 0 to 15 once under the name /var/int\_xy and 3 times 
under the name /var/int\_xyt. 

\vspace{10pt}
\$ bpls -latv g\_2x2\_2x2\_t3.bp

\vspace{10pt}
File info:

\vspace{10pt}
\parindent=7pt
of groups:     1

\vspace{10pt}
of variables:  11

\vspace{10pt}
of attributes: 7

\vspace{10pt}
time steps:    3 starting from 1

\vspace{10pt}
file size:     779 KB

\vspace{10pt}
bp version:    1

\vspace{10pt}
endianness:    Little Endian

\vspace{22pt}
\parindent=0pt
Group genarray:

\vspace{10pt}
\parindent=7pt
integer    /dimensions/X             scalar = 4 

\vspace{10pt}
integer    /dimensions/Y             scalar = 4 

\vspace{10pt}
integer    /info/nproc               scalar = 4 

\vspace{10pt}
string     /info/nproc/description   attr   = \texttt{"}Number of writers\texttt{"}

\vspace{10pt}
integer    /info/npx                 scalar = 2 

\vspace{10pt}
string     /info/npx/description     attr   = \texttt{"}Number of processors in 
x dimension\texttt{"}

\vspace{10pt}
integer    /info/npy                 scalar = 2 

\vspace{10pt}
string     /info/npy/description     attr   = \texttt{"}Number of processors in 
y dimension\texttt{"}

\vspace{10pt}
\textbf{integer    /var/int\_xy               \{4, 4\} = 0 / 15 }

\vspace{10pt}
string     /var/int\_xy/description   attr   = \texttt{"}2D array with 2D decomposition\texttt{"}

\vspace{10pt}
\textbf{integer    /var/int\_xyt              \{3, 4, 4\} = 0 / 15 }

\vspace{10pt}
string     /var/int\_xyt/description  attr   = \texttt{"}3D array with 2D decomposition 
with time in 3rd dimension\texttt{"}

\label{HToc144350180}

\vspace{22pt}
\leftskip=18pt
\parindent=0pt
{\color{color20} \textbf{Figure 21. bpls utility}}

\vspace{10pt}
\leftskip=0pt
The content of /var/int\_xy can be dumped with

\vspace{10pt}
\$ bpls  g\_2x2\_2x2\_t3.bp -d -n 4 var/int\_xy

\vspace{10pt}
\parindent=7pt
integer    /var/int\_xy    \{4, 4\}

\vspace{10pt}
\parindent=14pt
(0,0)    0 1 2 3 

\vspace{10pt}
(1,0)    4 5 6 7 

\vspace{10pt}
(2,0)    8 9 10 11 

\vspace{10pt}
(3,0)    12 13 14 15

\vspace{22pt}
\parindent=0pt
The ``central'' 2x2 subset of /var/int\_xy can be dumped with

\vspace{10pt}
\$ bpls g\_2x2\_2x2\_t3.bp -d -s \texttt{"}1,1\texttt{"} -c \texttt{"}2,2\texttt{"} 
-n 2 var/int\_xy

\vspace{10pt}
\parindent=7pt
integer    /var/int\_xy    \{4, 4\}

\vspace{10pt}
\parindent=14pt
slice (1:2, 1:2)

\vspace{10pt}
(1,1)    5 6 

\vspace{10pt}
(2,1)    9 10

\vspace{22pt}
\parindent=0pt
The last element of /var/int\_xyt for each timestep can be dumped with

\vspace{10pt}
\$ bpls g\_2x2\_2x2\_t3.bp -d -s \texttt{"}0,-1,-1\texttt{"} -c \texttt{"}-1,1,1\texttt{"} 
-n 1 var/int\_xyt

\vspace{10pt}
\parindent=7pt
integer    /var/int\_xyt   \{3, 4, 4\}

\vspace{10pt}
\parindent=14pt
slice (0:2, 3:3, 3:3)

\vspace{10pt}
(0,3,3)    15 

\vspace{10pt}
(1,3,3)    15 

\vspace{10pt}
(2,3,3)    15\label{HToc182553429}

\vspace{10pt}
\subsection*{{\large 9.4 }{\large \textbf{bpdump}}}

\vspace{10pt}
\parindent=0pt
The bpdump utility enables users to examine the contents of a bp file more closely 
to the actual BP format than with bpls and to display all the contents or selected 
variables in the format on the standard output. Each writing process' output is 
printed separately. 

\vspace{10pt}
\parindent=3pt
It dumps the bp file content, including the indexes for all the process groups, 
variables, and attributes, followed by the variables and attributes list of individual 
process groups (see Figure 22 ).

\vspace{22pt}
\parindent=0pt
bpdump [-d var\textbar{}--dump var] \texttt{<}filename\texttt{>}

\vspace{10pt}
========================================================

\vspace{10pt}
Process Groups Index:

\vspace{10pt}
Group: temperature

\vspace{10pt}
\parindent=28pt
Process ID: 0

\vspace{10pt}
Time Name:

\vspace{10pt}
\parindent=57pt
Time: 1

\vspace{10pt}
\parindent=28pt
Offset in File: 0

\vspace{10pt}
\parindent=0pt
========================================================

\vspace{10pt}
Vars Index:

\vspace{10pt}
Var (Group) [ID]: /NX (temperature) [1]

\vspace{10pt}
\parindent=28pt
Datatype: integer

\vspace{10pt}
Vars Characteristics: 20

\vspace{10pt}
\parindent=0pt
Offset(46)              Value(10)

\vspace{10pt}
Var (Group) [ID]: /size (temperature) [2]

\vspace{10pt}
\parindent=28pt
Datatype: integer

\vspace{10pt}
Vars Characteristics: 20

\vspace{10pt}
\parindent=0pt
Offset(77)              Value(20)...

\vspace{22pt}
Var (Group) [ID]: /rank (temperature) [3]

\vspace{10pt}
\parindent=28pt
Datatype: integer

\vspace{10pt}
Vars Characteristics: 20

\vspace{10pt}
\parindent=86pt
Offset(110)             Value(0)...

\vspace{22pt}
\parindent=0pt
Var (Group) [ID]: /temperature (temperature) [4]

\vspace{10pt}
\parindent=28pt
Datatype: double

\vspace{10pt}
Vars Characteristics: 20

\vspace{10pt}
\parindent=86pt
Offset(143)             Min(1.000000e-01)               Max(9.100000e+00)      
         Dims (l:g:o): (1:20:0,10:10:0)...

\vspace{22pt}
\parindent=0pt
========================================================

\vspace{10pt}
Attributes Index:

\vspace{10pt}
Attribute (Group) [ID]: /recorded-date (temperature) [5]

\vspace{10pt}
\parindent=28pt
Datatype: string

\vspace{10pt}
Attribute Characteristics: 20

\vspace{10pt}
\parindent=0pt
Offset(363)             Value(Sep-19-2008)...

\label{HRef119579247}\label{HToc144350181}

\vspace{34pt}
\leftskip=18pt
{\color{color20} \textbf{Figure 22. bpdump utility\label{HToc84890281}\label{HToc212016657}\label{HToc212016899}\label{HToc182553430}}}
