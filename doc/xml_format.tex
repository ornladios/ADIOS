\vspace{10pt}
%\section*{{\Large \textbf{5 XML Config File Format\label{HToc82067518}}}}
\section{XML Config File Format}

\vspace{24pt}
\subsection*{{\large 5.1 }{\large \textbf{ \label{HToc84890241}\label{HToc212016617}\label{HToc212016859}\label{HToc182553364}Overview}}}

\vspace{10pt}
\leftskip=0pt
XML is designed to allow users to store as much metadata as they can in an external 
configuration file. Thus the scientific applications are less polluted and require 
less effort to be verified again.

\vspace{10pt}
First, we present the XML template. Second, we demonstrate how to construct the 
XML file from the user's own source code. Third, we note how to troubleshoot and 
debug the errors in the file.  

\vspace{10pt}
Abstracting metadata, data type, and dimensions from the source code into an XML 
file gives users more flexibility to annotate the arrays or variables and centralizes 
the description of all the data structures, which in return, allows I/O componentization 
for different implementation of transport methods. By cataloguing the data types 
externally, we have an additional documentation source as well as a way to easily 
validate the write calls compared with the read calls without having to decipher 
the data reorganization or selection code that may be interspersed with the write 
calls. It is useful that the XML name attributes are just strings. The only restrictions 
for their content are that if the item is to be used in a dataset dimension, it 
must not contain commas and must contain at least one non-numeric character. This 
is useful for incorporating expressions as various array dimensions elements. Figure 
3 illustrates the corresponding XML configuration for the example we demonstrated 
in Figure 1. 

\vspace{10pt}
At a minimum, a configuration document must declare an adios-config element. It 
serves as a container for other elements; as such, it MUST be used as the root 
element. The expected children in any order would be adios-group, method, and buffer. 
The main elements of the xml file format are of the format 

\vspace{10pt}
\texttt{<}element-name attr1 attr2 ...\texttt{>}


\vspace{22pt}
\leftskip=28pt
\texttt{<}adios-config\texttt{>}

\vspace{10pt}
\leftskip=45pt
\texttt{<}adios-group\texttt{>}

\vspace{10pt}
\leftskip=63pt
\texttt{<}var /\texttt{>}......

\vspace{22pt}
\texttt{<}attribute /\texttt{>}......

\vspace{22pt}
\leftskip=45pt
\texttt{<}/adios-group\texttt{>}...

\vspace{22pt}
\texttt{<}method\texttt{>}......

\vspace{22pt}
\texttt{<}buffer\texttt{>}

\vspace{10pt}
\leftskip=28pt
\texttt{<}/adios-config\texttt{>}

\label{HToc144350162}\label{HRef119578307}

\vspace{22pt}
\leftskip=18pt
{\color{color20} \textbf{Figure 3. Example XML configuration\label{HToc82067519}\label{HToc84890242}\label{HToc212016618}\label{HToc212016860}\label{HToc182553365}}}

\vspace{10pt}
\subsection*{{\large 5.2 }{\large \textbf{adios-group}}}

\vspace{10pt}
\leftskip=0pt
The adios-group element represents a container for a list of variables that share 
the common I/O pattern as stated in the basic concepts of ADIOS in the first chapter. 
In this case, the group domain division logically corresponds to the different 
functions of output in scientific applications, such as restart, diagnosis, and 
snapshot. Depending on the different applications, adios-group can occur as many 
times as is needed. \label{HToc84890243}\label{HToc212016619}\label{HToc212016861}\label{HToc182553366}

\vspace{10pt}
\subsubsection*{{\large \textbf{5.2.1 Declaration}}}

\vspace{10pt}
The following example illustrates how to declare an adios group inside an XML file. 
First we start with adios-group as our tag name, which is case insensitive. It 
has an indispensable attribute called ``name,'' whose value is usually defined 
as a descriptive string indicating the function of the group. In this case, the 
string is called ``restart,'' because the files into which this group is written 
are used as checkpoints. The second attribute ``host-language'' indicates the language 
in which this group's I/O operations are written. The value of attribute ``coordination-communicator'' 
is used to coordinate the operations on a shared file accessed by multiple processes 
in the same communicator domain. ``Coordination-var'' provides the ability to use 
the user-defined variable, for example mype, rather than an MPI communicator for 
file coordination. 

\vspace{10pt}
\texttt{<}\textbf{adios-group} \textbf{name}=``restart'' 

\vspace{10pt}
\leftskip=72pt
host-language=``C''

\vspace{10pt}
coordination-communicator=''comm''

\vspace{10pt}
coordination-var=''mype''

\vspace{10pt}
time-index=''iter''/\texttt{>}

\vspace{10pt}
\leftskip=0pt
Required:

\vspace{10pt}
\leftskip=18pt
\parindent=3pt
name---containing a descriptive string to name the group

\vspace{10pt}
\leftskip=0pt
\parindent=0pt
Optional:

\vspace{10pt}
\leftskip=18pt
\parindent=3pt
host-language---language in which the source code for group is written

\vspace{10pt}
coordination-communicator---MPI-IO writing to a shared file

\vspace{10pt}
coordination-var---coordination variables for non-MPI methods, such as Datatap 
method

\vspace{10pt}
time-index---time attribute variable\label{HToc82067520}\label{HToc84890244}\label{HToc212016620}\label{HToc212016862}\label{HRef278369587}\label{HToc182553367}

\vspace{10pt}
\subsubsection*{{\large \textbf{5.2.2 Variables}}}

\vspace{10pt}
\leftskip=0pt
\parindent=0pt
The nested variable element ``var'' for adios\_group, which can be either an array 
or a primitive data type, is determined by the dimension attribute provided. 

\vspace{10pt}
\textbf{5.2.2.1 Declaration}

\vspace{10pt}
The following is an example showing how to define a variable in the XML file. 

\vspace{10pt}
\texttt{<}\textbf{var} \textbf{name}=``z-plane ion particles'' 

\vspace{10pt}
\leftskip=27pt
gwrite=``zion''

\vspace{10pt}
gread=''zion\_read''

\vspace{10pt}
\textbf{type}=''adios\_real''

\vspace{10pt}
dimensions=''7,mimax''

\vspace{10pt}
read=''yes''/\texttt{>}

\vspace{22pt}
\leftskip=0pt
\textbf{5.2.2.2 Attribute list}

\vspace{10pt}
The attributes associated with var element  as follows: 

\vspace{10pt}
Required:

\vspace{10pt}
\leftskip=18pt
\parindent=3pt
name - the string name of variable stored in the output file

\vspace{10pt}
type - the data type of the variable

\vspace{10pt}
\leftskip=0pt
\parindent=0pt
Optional: 

\vspace{10pt}
\leftskip=72pt
\parindent=-50pt
gwrite - the value will be used in the python scripts to generate adios\_write 
routines; the default value will be the same as attribute \textit{name} if gwrite 
is not defined.

\vspace{10pt}
\leftskip=67pt
\parindent=-45pt
gread - the value will be used in the python scripts to generate adios\_read routines' 
the default value will be the same as attribute \textit{name} if  gread is not 
defined.

\vspace{10pt}
\leftskip=58pt
\parindent=-36pt
path - HDF-5-style path for the element or path to the HDF-5 group or data item 
to which this attribute is attached.  The default value is ``/''.

\vspace{10pt}
\leftskip=94pt
\parindent=-72pt
dimensions - a comma-separated list of numbers and/or names that correspond to 
integer var elements determine the size of this item. If not specified, the variable 
is scalar.

\vspace{10pt}
\leftskip=58pt
\parindent=-36pt
read - value is either \textit{yes} or \textit{no}; in the case of no, the adios\_read 
routine will not be generated for this var entry. If undefined, the default value 
will be yes. \label{HToc82064271}\label{HToc82067521}\label{HToc84890245}\label{HToc212016621}\label{HToc212016863}\label{HRef278897265}\label{HToc182553368}

\vspace{10pt}
\subsubsection*{{\large \textbf{5.2.3 Attributes}}}

\vspace{10pt}
\leftskip=0pt
\parindent=0pt
The attribute element for adios\_group provides the users with the ability to specify 
more descriptive information about the variables or group. The attributes can be 
defined in both static or dynamic fashions. \label{HToc82064272}

\vspace{10pt}
\textbf{5.2.3.1 Declaration}

\vspace{10pt}
The static type of attributes can be defined as follows:

\vspace{10pt}
\texttt{<}\textbf{attribute} name=``experimental date'' 

\vspace{10pt}
\leftskip=27pt
\parindent=43pt
path=``/zion''

\vspace{10pt}
value=''Sep-19-2008''

\vspace{10pt}
\parindent=0pt
type=''adios\_real''/\texttt{>}

\vspace{22pt}
\leftskip=0pt
If an attribute has dynamic value that is determined by the runtime execution of 
the program, it can be specified as follows:

\vspace{10pt}
\texttt{<}\textbf{attribute} name=``experimental date'' 

\vspace{10pt}
\leftskip=27pt
\parindent=43pt
path=``/zion''

\vspace{10pt}
var=''time''/\texttt{>}

\vspace{22pt}
\leftskip=0pt
\parindent=0pt
where var ``time'' need to be defined in the same adios-group.

\vspace{10pt}
\textbf{5.2.3.2 Attribute list}

\vspace{10pt}
Required:\label{HToc82067522}

\vspace{10pt}
\leftskip=18pt
\parindent=3pt
name -  name of the attribute

\vspace{10pt}
path - hierarchical path inside the file for the attribute

\vspace{10pt}
value - attribute has static value of the attribute, mutually exclusive with the 
attribute \textit{var}

\vspace{10pt}
type - string or numeric type, paired with attribute \textit{value}, in other words,, 
mutually exclusive with the attribute \textit{var} also

\vspace{10pt}
var - attribute has dynamic value that is defined by a variable in \textit{var\label{HToc84890246}\label{HToc212016622}\label{HToc212016864}\label{HToc182553369}}

\vspace{10pt}
\subsubsection*{{\large \textbf{5.2.4 Gwrite src}}}

\vspace{10pt}
\leftskip=0pt
\parindent=0pt
The element \texttt{<}Gwrite src=\texttt{"}...\texttt{"}\texttt{>} is unlike \texttt{<}var\texttt{>} 
or \texttt{<}attribute\texttt{>}, which are parsed and stored in the internal file 
structure in ADIOS. The element \texttt{<}gwrite\texttt{>} only affects the execution 
of python scripts (see Chap. 10). Any content (usually comments, conditional statements, 
or loop statements) in the value of attribute ``src'' is copied identically into 
generated pre-processing files. Declaration

\vspace{10pt}
\texttt{<}gwrite src=''    ''/\texttt{>}

\vspace{10pt}
Required:

\vspace{10pt}
\leftskip=18pt
\parindent=3pt
src -  any statement that needs to be added into the source code. This code will 
be inserted into the source code, and must be able to be compiled in the host language, 
C or Fortran. \label{HToc84890247}\label{HToc212016623}\label{HToc212016865}\label{HRef278371183}\label{HToc182553370}

\vspace{10pt}
\subsubsection*{{\large \textbf{5.2.5 Global arrays}}}

\vspace{10pt}
\leftskip=0pt
\parindent=0pt
The \textbf{global-bounds} element is an optional nested element for the adios-group. 
It specifies the global space and offsets within that space for the enclosed variable 
elements. In the case of writing to a shared file, the global-bounds information 
is recorded in BP file and can be interpreted by converters or other postprocessing 
tools or used to write out either HDF5 or NetCDF files by using PHDF5 or the PnetCDF 
method.\label{HToc84890248}\label{HToc212016624}\label{HToc212016866}\label{HRef278371225}\label{HToc182553371}

\vspace{10pt}
\subsubsection*{{\large \textbf{5.2.6 Time-index}}}

\vspace{10pt}
ADIOS allows a dataset to be expanded in the space domain given by global bounds 
and in time domain. It is very common for scientific applications to write out 
a monitoring file at regular intervals. The file usually contains a group of time-based 
variables that have undetermined dimensional value on the time axis. ADIOS is Similar 
to NetCDF in that it accumulates the time-index in terms of the number of records, 
which theoretically can be added to infinity.

\vspace{10pt}
If any of variables in an adios group are time based, they can be marked out by 
adding the time-index variable as another dimension value. 

\vspace{10pt}
\textbf{5.2.6.1 Declaration}

\vspace{10pt}
\texttt{<}\textbf{global-bounds} \textbf{dimensions}=``nx\_g, ny\_g'' \textbf{offsets}=``nx\_o,0''/\texttt{>}...

\vspace{10pt}
\parindent=21pt
variable declarations ...

\vspace{10pt}
\parindent=0pt
\texttt{<}/\textbf{global-bounds\texttt{>}}

\vspace{10pt}
Required:

\vspace{10pt}
\leftskip=18pt
\parindent=3pt
dimensions - the dimension of global space

\vspace{10pt}
offsets - the offset of the data set in global space

\vspace{10pt}
\leftskip=0pt
\parindent=0pt
Any variables used in the global-bounds element for dimensions or offsets declaration 
need to be defined in the same adios-group as either variables or attributes. 

\vspace{10pt}
For detailed global arrays use, see the example illustrated in Section 12.8.

\vspace{10pt}
\textbf{Changing I/O Without Changing Source:} The method element provides the 
hook between the adios-group and the transport methods. The user employs a different 
transport method simply by changing the method attribute of the method element. 
If more than one method element is provided for a given group, each element will 
be invoked in the order specified. This neatly gives triggering opportunities for 
workflows. To trigger a workflow once the analysis data set has been written to 
disk, the user makes two element entries for the analysis adios-group. The first 
indicates how to write to disk, and the second performs the trigger for the workflow 
system. No recompilation, relinking, or any other code changes are required for 
any of these changes to the XML file.\label{HToc82067523}\label{HToc84890250}\label{HToc212016626}\label{HToc212016868}\label{HToc182553372}

\vspace{10pt}
\subsubsection*{{\large \textbf{5.2.7 Declaration}}}

\vspace{10pt}
The transport element is used to specify the mapping of an I/O transport method, 
including optional initialization parameters, to the respective adios-group. There 
are two major attributes required for the method element: 

\vspace{10pt}
\texttt{<}\textbf{transport} \textbf{group}=``restart'' 

\vspace{10pt}
\leftskip=54pt
\textbf{method}=``MPI''

\vspace{10pt}
priority=''1''

\vspace{10pt}
iteration=''100''/\texttt{>}

\vspace{10pt}
\leftskip=0pt
Required:

\vspace{10pt}
\leftskip=18pt
\parindent=3pt
group - corresponds to an adios-group specified earlier in the file.

\vspace{10pt}
method - a string indicating a transport method to use with the associated adios-group

\vspace{10pt}
\leftskip=0pt
\parindent=0pt
Optional: 

\vspace{10pt}
\leftskip=72pt
\parindent=-50pt
priority- a numeric priority for the I/O method to better schedule this write with 
others that may be pending currently

\vspace{10pt}
\leftskip=85pt
\parindent=-85pt
base-path-the root directory to use when writing to disk or similar purposes

\vspace{10pt}
\parindent=-63pt
iterations- a number of iterations between writes of this group used to gauge how 
quickly this data should be evacuated from the compute node\label{HToc82067524}\label{HToc84890251}\label{HToc212016627}\label{HToc212016869}\label{HToc182553373}

\vspace{10pt}
\subsubsection*{{\large \textbf{5.2.8 Methods list}}}

\vspace{10pt}
\leftskip=0pt
\parindent=0pt
As the componentization of the IO substrate, ADIOS supports a list of transport 
methods, described in Section 6:

\vspace{10pt}
\ensuremath{\Sigma} NULL

\vspace{10pt}
\ensuremath{\Sigma} POSIX

\vspace{10pt}
\ensuremath{\Sigma} MPI

\vspace{10pt}
\ensuremath{\Sigma} MPI-LUSTRE

\vspace{10pt}
\ensuremath{\Sigma} MPI-AMR

\vspace{10pt}
\ensuremath{\Sigma} PHDF5

\vspace{10pt}
\ensuremath{\Sigma} NC4 (NETCDF4)

\vspace{10pt}
\ensuremath{\Sigma} NSSI

\vspace{10pt}
\ensuremath{\Sigma} DATATAP 

\vspace{10pt}
\ensuremath{\Sigma} DART\label{HToc82067525}\label{HToc84890252}\label{HToc212016628}\label{HToc212016870}\label{HRef278900114}\label{HToc182553374}

\vspace{10pt}
\subsection*{{\large 5.3 }{\large \textbf{Buffer specification}}}

\vspace{10pt}
The buffer element defines the attributes for internal buffer size and creating 
time that apply to the whole application (Figure 4). The attribute allocate-time 
is identified as being either{\small  ``}now'' or ``oncall'' to indicate when the 
buffer should be allocated. An ``oncall'' attribute waits until the programmer 
decides that all memory needed for calculation has been allocated. It then calls 
upon ADIOS to allocate buffer. There are two alternative attributes for users to 
define the buffer size: MB and free-memory-percentage. \label{HToc84890253}\label{HToc212016629}\label{HToc212016871}\label{HToc182553375}

\vspace{10pt}
\subsubsection*{{\large \textbf{5.3.1 Declaration}}}

\vspace{10pt}
\texttt{<}\textbf{buffer} \textbf{size-MB}=``100'' 

\vspace{10pt}
\leftskip=45pt
\textbf{allocate-time}=``now'' /\texttt{>}

\vspace{22pt}
\leftskip=0pt
Required:

\vspace{10pt}
\leftskip=76pt
\parindent=-58pt
size-MB - the user-defined size of  buffer in megabytes. ADIOS can at most allocate 
from compute nodes. It is exclusive with free-memory-percentage.

\vspace{10pt}
free-memory percentage - the user-defined percentage from 0 to 100\% of free memory 
available on the machine. It is exclusive with size-MB.

\vspace{10pt}
\leftskip=18pt
\parindent=3pt
allocate-time - indicates when the buffer should be allocated\label{HToc84890254}\label{HToc212016630}\label{HToc212016872}\label{HToc182553376}

\vspace{10pt}
\subsection*{{\large 5.4 }{\large \textbf{Enabling Histogram}}}

\vspace{10pt}
\leftskip=0pt
\parindent=0pt
ADIOS 1.2 has the ability to {\color{color01} compute a histogram of the given 
variable's data values at write time}. This is specified via the \textbf{\texttt{<}analysis\texttt{>}} 
tag in the XML file. The parameters \texttt{"}\textbf{adios-group}\texttt{"} and 
\texttt{"}\textbf{var}\texttt{"} specify the variable for which the histogram is 
to be performed. \texttt{"}\textbf{var}\texttt{"} is the name of the variable and 
\texttt{"}\textbf{adios-group}\texttt{"} is the name of the adios group to which 
the variable belongs to. \label{HToc182553377}

\vspace{10pt}
\subsubsection*{{\large \textbf{5.4.1 Declaration}}}

\vspace{10pt}
The histogram binning intervals can be input in two ways via the XML file:

\vspace{10pt}
\ensuremath{\Sigma} By listing the break points as a list of comma separated values 
in the parameter \texttt{"}break-points\texttt{"} 

\vspace{10pt}
\leftskip=99pt
\parindent=-59pt
\texttt{<}\textbf{analysis adios-group}=\texttt{"}temperature\texttt{"}    \textbf{var}=\texttt{"}temperature\texttt{"}

\vspace{10pt}
\parindent=-63pt
\textbf{break-points}=\texttt{"}0, 100, 200, 300\texttt{"} /\texttt{>}

\vspace{10pt}
\leftskip=0pt
\parindent=0pt
\ensuremath{\Sigma} By specifying the boundaries of the breaks, and the number 
of intervals between variable's min and max values

\vspace{10pt}
\leftskip=99pt
\parindent=-63pt
\texttt{<}\textbf{analysis adios-group}=\texttt{"}temperature\texttt{"}    \textbf{var}=\texttt{"}temperature\texttt{"}

\vspace{10pt}
\textbf{min}=\texttt{"}0\texttt{"}    \textbf{max}=\texttt{"}300\texttt{"}    \textbf{count}=\texttt{"}3\texttt{"}/\texttt{>}

\vspace{10pt}
\leftskip=0pt
\parindent=0pt
Both inputs create the bins (-Inf, 0), [0, 100), [100, 200), [200, 300), [300, 
Inf). For this example, the final set of frequencies for these 5 binning intervals 
will be calculated.

\vspace{10pt}
Required:

\vspace{10pt}
\leftskip=76pt
\parindent=-54pt
adios-group - corresponds to an adios-group specified earlier in the file.

\vspace{10pt}
var - corresponds to a variable in adios-group specified earlier in the file.

\vspace{10pt}
\leftskip=0pt
\parindent=0pt
Optional:

\vspace{10pt}
\leftskip=36pt
\parindent=-18pt
break-points - list of comma separated values \textbf{sorted} in ascending order

\vspace{10pt}
\parindent=-3pt
min  - minimum value of the binning boundary

\vspace{10pt}
\parindent=-18pt
max - maximum value of the binning boundary 

\vspace{10pt}
(it should be greater than min)

\vspace{10pt}
count - number of break points between the min and max values 

\vspace{10pt}
\leftskip=0pt
\parindent=0pt
A valid set of binning intervals must be provided either by specifying \texttt{"}min,\texttt{"} 
\texttt{"}max,\texttt{"} and \texttt{"}count\texttt{"} parameters or by providing 
the \texttt{"}break-points.\texttt{"} The intervals given under \texttt{"}break-points\texttt{"} 
will take precedence when calculating the histogram intervals, if \texttt{"}min,\texttt{"} 
\texttt{"}max,\texttt{"} and \texttt{"}count\texttt{"} as well as ``break-points'' 
are provided.\pagebreak{}\label{HToc182553378}

\vspace{10pt}
\subsection*{{\large \textbf{An Example XML file}}}


\vspace{22pt}
\texttt{<}\textbf{adios-config} host-language=\texttt{"}C\texttt{"}\texttt{>}

\vspace{22pt}
\parindent=14pt
\texttt{<}\textbf{adios-group} name=\texttt{"}temperature\texttt{"} coordination-communicator=\texttt{"}comm\texttt{"}\texttt{>}

\vspace{10pt}
\parindent=28pt
\texttt{<}var name=\texttt{"}NX\texttt{"} type=\texttt{"}integer\texttt{"}/\texttt{>}

\vspace{10pt}
\texttt{<}var name=\texttt{"}t\texttt{"} type=\texttt{"}double\texttt{"} dimensions=\texttt{"}NX\texttt{"}/\texttt{>}

\vspace{10pt}
\parindent=57pt
\texttt{<}attribute name=\texttt{"}recorded date\texttt{"} path=\texttt{"}/\texttt{"} 
value=\texttt{"}Sep 19, 2008\texttt{"} type=\texttt{"}string\texttt{"}/\texttt{>}

\vspace{22pt}
\parindent=28pt
{\color{color26} \texttt{<}!-- conditional writing of a variable --\texttt{>}}

\vspace{10pt}
\parindent=0pt
\texttt{<}gwrite src=\texttt{"}if (want\_x) \{\texttt{"} /\texttt{>}

\vspace{10pt}
\parindent=43pt
\texttt{<}var name=\texttt{"}x\texttt{"} type=\texttt{"}integer\texttt{"} dimensions=\texttt{"}NX\texttt{"}/\texttt{>}

\vspace{10pt}
\parindent=0pt
\texttt{<}gwrite src=\texttt{"}\}\texttt{"} /\texttt{>}

\vspace{22pt}
\parindent=14pt
\texttt{<}/\textbf{adios-group}\texttt{>}

\vspace{22pt}
\texttt{<}\textbf{method} group=\texttt{"} temperature \texttt{"} method=\texttt{"}MPI\texttt{"}/\texttt{>}

\vspace{22pt}
\texttt{<}\textbf{buffer} size-MB=\texttt{"}1\texttt{"} allocate-time=\texttt{"}now\texttt{"}/\texttt{>}

\vspace{10pt}
\texttt{<}\textbf{analysis }adios-group=\texttt{"}temperature\texttt{"} var=\texttt{"}t\texttt{"} 
break-points=\texttt{"}0, 100, 200, 300\texttt{"}/\texttt{>}

\vspace{22pt}
\parindent=0pt
\texttt{<}/\textbf{adios-config}\texttt{>}

\label{HRef119578444}\label{HToc144350163}

\vspace{22pt}
\leftskip=18pt
{\color{color20} \textbf{Figure 4. Example XML file.\label{HToc84890255}\label{HToc212016631}\label{HToc212016873}\label{HRef212367123}\label{HToc82067526}\label{HRef278374025}\label{HRef278895952}\label{HToc182553379}}}
